\documentclass[11pt]{article}

%
% Packages
%

\usepackage[utf8]{inputenc}
\usepackage[T1]{fontenc}
\usepackage{lmodern}
\usepackage{listings}
\usepackage{amsmath}
\allowdisplaybreaks
\usepackage{amsfonts}
\usepackage{amstext}
\usepackage{amssymb}
\usepackage{amsthm}
\usepackage{empheq}
\usepackage{cases}
\usepackage{anyfontsize}
\usepackage{enumitem}
\usepackage{pdfpages}
\usepackage{fourier}	% Style
\usepackage{bm}
\usepackage{epstopdf}
\usepackage{lipsum}
\usepackage[top=3cm, bottom=3cm, left=2cm, right=2cm, scale=0.75]{geometry}	% Set the margins
\usepackage{fancyhdr}
\usepackage[letterspace=150]{microtype}
\usepackage{textcomp}
\usepackage{gensymb}
\usepackage{booktabs}
\usepackage{amsmath,etoolbox}
\usepackage{mathtools}
\usepackage{anyfontsize}
\usepackage{graphicx}
\usepackage{epstopdf}
\usepackage{float}
\usepackage{subfig}
\usepackage[labelfont=bf,labelsep=period,font=small]{caption}
\usepackage{newunicodechar}
\usepackage{nicefrac}	% For diagonal fractions
\usepackage{bbm}
\usepackage{csvsimple}

% Set header and footer
\usepackage{fancyhdr}

% Packages needed for tables
\usepackage{longtable}
\usepackage{multicol}
\usepackage{multirow}
\usepackage{array}

\PassOptionsToPackage{hyphens}{url}\usepackage{hyperref}
\usepackage{breakurl}
\usepackage{url}

% To put footnotes at the bottom of the page
\usepackage[bottom]{footmisc}

\usepackage{empheq}

\usepackage{tikz}

% Default fixed font does not support bold face
\DeclareFixedFont{\ttb}{T1}{txtt}{bx}{n}{10.25} % for bold
\DeclareFixedFont{\ttm}{T1}{txtt}{m}{n}{10.25}  % for normal

% Custom colors
\usepackage{color}
\definecolor{deepblue}{rgb}{0,0,0.5}
\definecolor{deepred}{rgb}{0.6,0,0}
\definecolor{deepgreen}{rgb}{0,0.5,0}

% To insert code snippets
\usepackage{listings}

% For argmin
\DeclareMathOperator*{\argmin}{arg\,min}

% To insert verbatim within a command
\usepackage{fancyvrb}

% For pseudocode
\usepackage[section]{algorithm}
\usepackage{algpseudocode}

\usepackage[many]{tcolorbox}

\usepackage{stackengine}

% Equations enumeration options
\numberwithin{equation}{section}

% Set interline
\usepackage{setspace}

% Path to images
\graphicspath{{./img/}}


%
% Definitions
%

\DeclarePairedDelimiter\abs{\lvert}{\rvert}
\makeatletter
\let\oldabs\abs
\def\abs{\@ifstar{\oldabs}{\oldabs*}}

% Theorem and definition environment
\theoremstyle{theorem}
\newtheorem{theorem}{Theorem}[section]
\theoremstyle{definition}
\newtheorem{definition}{Definition}[section]
\theoremstyle{remark}
\newtheorem{remark}{Remark}[section]
\theoremstyle{proposition}
\newtheorem{proposition}{Proposition}[section]

% To enumerate the equations and the figures according to the section they are in
%\numberwithin{equation}{section}
\numberwithin{figure}{section}

% To modify the space between figure and caption
%\setlength{\abovecaptionskip}{-4pt}
%\setlength{\belowcaptionskip}{3pt}

\renewcommand{\textfraction}{0.1}
\renewcommand{\topfraction}{0.9}

\makeatletter
	\renewcommand*\l@figure{\@dottedtocline{1}{1em}{3.2em}}
\makeatother

% Define norm symbol
\newcommand{\norm}[1]{\left\lVert#1\right\rVert}

% Define mod symbol
\newcommand{\Mod}[1]{\ (\mathrm{mod}\ #1)}

% Aliases for \boldsymbol and \widetilde
\newcommand{\wt}[1]{\widetilde{#1}}
\newcommand{\bg}[1]{\boldsymbol{#1}}

% Redefine \Require and \Ensure for algorithm environment
\renewcommand{\algorithmicrequire}{\textbf{Input:}}
\renewcommand{\algorithmicensure}{\textbf{Output:}}

% Make \big| adapt to the context
\makeatletter
\let\amstexbig\big
\def\newbig#1{%
  \ifx#1|%
    \expandafter\@firstoftwo
  \else
    \expandafter\@secondoftwo
  \fi
  {\big@bar}%
  {\amstexbig{#1}}%
}
\AtBeginDocument{\let\big\newbig}
\def\big@bar{\bBigg@{1.1}|}
\makeatother

% Define the do-while loop
\algdef{SE}[DOWHILE]{DoWhile}{EndDoWhile}{\algorithmicdo}[1]{\algorithmicwhile\ #1}


%
% Document
%

\begin{document}
	
	\textbf{Significance and novelty of this paper} \\

	We present a non-intrusive reduced basis (RB) method for parametrized steady-state partial differential equations (PDEs). The method extracts a reduced basis from a collection of high-fidelity solutions (\emph{snapshots}) via a proper orthogonal decomposition (POD) and resorts to neural networks (NNs), particularly multi-layer perceptrons (MLPs), to recover the coefficients of the reduced model. A single MLP is used for each state variable. The training of each MLP is carried out in the offline phase, relying upon the Levenberg-Marquardt algorithm and the latin hypercube sampling to identify the optimal amount of training patterns and neurons to avoid overfitting. This enables a complete offline-online decoupling, leading to an RB method (denoted POD-NN) suitable also for general nonlinear problems with a non-affine dependence on the parameters. 

	Numerical studies are presented for the one-dimensional and two-dimensional nonlinear Poisson equation and the lid-driven cavity problem for the incompressible steady Navier-Stokes equations; both physical and geometrical paramatrizations are considered. Several results show the reliability and efficiency of the proposed method. Particularly, we can attain the same accuracy enabled by a standard RB strategy while reducing the run time during the online stage by three order of magnitudes. This has to be ascribed to the fast evaluation of MLPs. 

	The secret hope behind the novel use of MLPs in the RB framework is that the number of samples (i.e., snapshots) required in the training phase properly scales with the dimension of the parameter space. Indeed, standard interpolation procedures (e.g., cubic splines) may require a large amount of samples to enforce the nonlinear constraints which usually characterize reduced bases. This is numerically verified in the case of a Poisson problem involving three parameters. 

\end{document}
